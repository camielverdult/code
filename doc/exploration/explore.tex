\documentclass{article}
\usepackage{preamble}
\begin{document}
\title{Fundamentals of exploration in \GROOVE}
\author{Arend Rensink}
\date{August 2024}
\maketitle

\medskip\noindent
We globally assume a set of labels $A$.

\medskip\noindent 
A state space is a tuple $\cS=\tupof{S,\iota,{\ra},{\up},\Tdepth}$ with
\begin{itemize}
\item $S$ a set of states;
\item $\iota\in S$ the initial state;
\item ${\ra}\subseteq S\times A\times S$ a transition relation;
\item ${\up}\subseteq S$ a termination predicate;
\item ${\Tdepth}:S\to \natN$ a \emph{transient depth} (or \emph{transience}) function.
\end{itemize}
%
State $s\in S$ is called \emph{final} if $p\up$, \emph{transient} (denoted $\Trnt(s)$) if $\Tdepth(s)>0$ and \emph{stable} (denoted $\Stable(s)$) if $\Tdepth(s)=0$. There is a derived \emph{transaction relation}:
%
\[ s\ttrans s' \enspace\iffdef\enspace \Stable(s) \wedge \Stable(s') \wedge s\trans{}^+ s' \enspace. \]
%
State spaces are generated from a \emph{pseudo-state spaces}.

\medskip\noindent
A pseudo-state space is a tuple $\cP=\tupof{P,\iota,{\mapsto},{\goesto},{\up},\Tdepth}$ with
\begin{itemize}
\item $P$ a finite set of pseudo-states;
\item $\iota\in P$ the initial pseudo-state;
\item ${\step{}}\subseteq P\times A\times P$ a step relation;
\item ${\goesto}: P\times P$ an acyclic evolution relation;
\item ${\up} \subseteq P$ a termination predicate;
\item ${\Tdepth}:P\to \natN$ a transient depth function.
\end{itemize}
%
Pseudo-state $p\in P$ is called \emph{prime} (denoted $\Prime(p)$) if $p\ncomesfrom$, \emph{closed} (denoted $\Closed(p)$) if $p\ngoesto$ and \emph{open} (denoted $\Open(p)$) if it is not closed.
%
A pseudo-state space is \emph{well-formed} if it satisfies the following additional properties:
\begin{itemize}
\item Stepping is deterministic; i.e., $\step{}$ is a partial function from $P$ to $A\times P$;
\item Evolution is deterministic; i.e., $\goesto$ is a partial function from $P$ to $P$;
\item Evolution is injective; i.e., $\comesfrom$ is a partial function from $P$ to $P$;
\item All steps go from open to prime pseudo-states; i.e., $p\step{}q$ implies $\Open(p)$ and $\Prime(q)$;
\item All final pseudo-states are stable and closed; i.e., $p\up$ implies $\Stable(p)$ and $\Closed(p)$;
\item Stepping cannot decrease transience; i.e., $p\step{}q$ implies $\Tdepth(q)\geq \Tdepth(p)$;
\item Evolution cannot increase transience; i.e., $p\goesto q$ implies $\Tdepth(q)\leq \Tdepth(p)$.
\end{itemize}
%
From now on, we only deal with well-formed pseudo-states spaces. The \emph{prime of} and \emph{closure of} a pseudo-state $p$ are defined as
%
\begin{align*}
	\prm p & = q \quad \text{where $\Prime (q)$ and $q\goesto^* p$} \\
	\cls p & = q \quad \text{where $p\goesto^* q$ and $\Closed(q)$} \enspace.
\end{align*}
%
Note that these are well-defined because $P$ is finite and $\goesto$ is acyclic, deterministic and injective (implying that it is a union of finite chains of pseudo-states). A pseudo-state space effectively represents a state space in which the states are $\goesto$-related sets of pseudo-states --- which, as remarked above, are actually chains. Rather than formalising the relation in this way, however, we let a state be represented by the initial element of such a chain, i.e., by the (unique) prime pseudo-state.

\medskip\noindent
Given a pseudo-state space $\cP$ as above, a \emph{configuration} is a $\goesto$-left-closed set $C\subseteq P$ of pseudo-states such that, moreover, $\iota\in C$ and if $p\comesfrom\:\step q$ then $p\in C$ implies $q\in C$. (It follows that $P$ is itself a configuration of $\cP$.) Given a configuration $C$, we define $\cS_C= \tupof{S_C,\iota_C,\trans{}_C, \up_C, \Tdepth_C}$ as follows:
%
\begin{align*}
S_C & = \gensetof{\prm p}{p\in C} \\
\iota_C & = \iota \\
{\trans{}_C} & = \gensetof{(\prm p,a,q)}{p,q\in C, p\step a q} \\
\up_C & = \gensetof{\prm p}{p\in C, p\up} \\
\Tdepth_C & : p \mapsto \min \gensetof{\Tdepth(q)}{q\in C,p=\prm q} \enspace \text{for all $p\in S_C$} \enspace.
\end{align*}
%
The notion of closedness and stability are also extended to configurations (so as to coincide with their counterparts for $S_C$):
\begin{align*}
\Closed_C & = \gensetof{\prm p}{p\in C, p\ngoesto} \\
\Stable_C & = \gensetof{\prm p}{p\in C, \Tdepth(p)=0} \enspace.
\end{align*}
%
We define some further auxiliary notions.
%
\begin{itemize}
\item $\Done_C(s)$ for $s\in S_C$ expresses that $s$ as well as all its discovered $\trans{}_C$-successors are closed, up to and including the first stable state. It is defined as the smallest set such that
\[ \Done_C = \Closed_C\cap \bigl(\Stable_C \cup \gensetof{p\in S_C}{\forall p\trans{}_C q.\, q\in \Done_C}\bigr) \enspace.
\]

\item $\ETdepth_C(s)$ for $s\in S_C$ is the \emph{eventual transient depth} in $C$, meaning the minimum transient depth of $s$ and all its $\trans{}_C$-successors. It is defined by
%
\[ \ETdepth_C: p\mapsto \min\gensetof{\Tdepth(q)}{p\trans{}_C^* q} \text{ for all $p\in S_C$} \enspace. \]

\item $\EStable_C(s)$ for $s\in S_C$ expresses that $s$ is \emph{eventually stable} in $C$, meaning that it or one of its discovered $\trans{}_C$-successors is stable (i.e., has transient depth 0). It is defined by
%
\[ \EStable_C=\gensetof{p\in S_C}{\ETdepth_C(p)=0} \enspace. \]

\item $\Abs_C(s)$ for $s\in S_C$ expresses that $s$ is \emph{absent} in $C$, which is the case if it is done (all reachable stable states have been discovered) but not eventually stable (no reachable stable state has been discovered). It is defined by
\[ \Abs_C=\Done_C\setminus \EStable_C \enspace. \]
\end{itemize}
%
We want to construct $S_P$ incrementally by approaching $P$ through a sequence of configurations, starting with $\setof{\iota}$ and adding, at each iteration, an evolution $p\goesto p'$ with $p\in C$ and $p'\notin C$. The latter is defined by 
\[ C\oplus(p\goesto p') = C\cup \setof{p'} \cup \gensetof{q}{p\step a q} \enspace. \]
Incremental construction means that, if $D=C\oplus(p\goesto p')$, all components of $S_D$ can be constructed from $S_C$ and $\cP$. Indeed, we have
%
\begin{align*}
\trans{}_D & = {\trans{}_C} \cup\gensetof{(\prm p,a,q)}{p\step a q} \\
\up_D & = \up_C \cup \gensetof{p'}{p'\up} \\
\Tdepth_D & : s\mapsto\begin{cases}
\Tdepth(p') & \text{if $s=\prm p$} \\
\Tdepth(q) & \text{if $s=q\notin S_C$} \\
\Tdepth_C(s) & \text{otherwise}
\end{cases} \\
\Closed_D & = \Closed_C \cup\gensetof{\prm p}{p'\ngoesto} \\
\Stable_D & = \Stable_C \cup \gensetof{\prm p}{\Tdepth(p')=0} \enspace.
\end{align*}
%
For $\ttrans_C$, however, the case is less easy. For the purpose of this construction, we introduce several auxiliary data structures.


\end{document}
