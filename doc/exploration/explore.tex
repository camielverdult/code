\documentclass{article}
\usepackage{preamble}
\begin{document}
\title{Fundamentals of exploration in \GROOVE}
\author{Arend Rensink}
\date{August 2024}
\maketitle

\medskip\noindent
We globally assume a set of labels $A$.

\medskip\noindent 
A state space is a tuple $\tupof{S,{\ra},{\up},\Tdepth}$ with
\begin{itemize}
\item $S$ a set of states;
\item ${\ra}\subseteq S\times A\times S$ a transition relation;
\item ${\up}\subseteq S$ a termination predicate;
\item ${\Tdepth}:S\to \natN$ a \emph{transient depth} (or \emph{transience}) function.
\end{itemize}
%
State $s\in S$ is called \emph{final} if $p\up$, \emph{transient} (denoted $\Trnt(s)$) if $\Tdepth(s)>0$ and \emph{stable} (denoted $\Stable(s)$) if $\Tdepth(s)=0$.

\medskip\noindent State spaces are generated from a \emph{pseudo-state spaces}.

\medskip\noindent
A pseudo-state space is a tuple $\tupof{P,{\mapsto},{\goesto},{\up},\Tdepth}$ with
\begin{itemize}
\item $P$ a finite set of pseudo-states;
\item ${\step{}}\subseteq P\times A\times P$ a step relation;
\item ${\goesto}: P\times P$ a partial, acyclic evolution relation;
\item ${\up} \subseteq P$ a termination predicate;
\item ${\Tdepth}:S\to \natN$ a \emph{transient depth} function.
\end{itemize}
%
Pseudo-state $p\in P$ is called \emph{prime} (denoted $\Prime(p)$) if $p\ncomesfrom$, \emph{closed} (denoted $\Closed(p)$) if $p\ngoesto$ and \emph{open} (denoted $\Open(p)$) if it is not closed.
%
A pseudo-state space is \emph{well-formed} if it satisfies the following additional properties:
\begin{itemize}
\item Stepping is deterministic; i.e., $\step{}$ is a partial function from $P$ to $A\times P$;
\item Evolution is deterministic; i.e., $\goesto$ is a partial function from $P$ to $P$;
\item Evolution is injective; i.e., $\comesfrom$ is a partial function from $P$ to $P$;
\item All steps go from open to prime pseudo-states; i.e., $p\step{}q$ implies $\Open(p)$ and $\Prime(q)$;
\item All final pseudo-states are stable and closed; i.e., $p\up$ implies $\Stable(p)$ and $\Closed(p)$;
\item Stepping cannot decrease transience; i.e., $p\step{}q$ implies $\Tdepth(q)\geq \Tdepth(p)$;
\item Evolution cannot increase transience; i.e., $p\goesto q$ implies $\Tdepth(q)\leq \Tdepth(p)$.
\end{itemize}
From now on, we only deal with well-formed pseudo-states spaces. The \emph{prime of} and \emph{closure of} a pseudo-state $p$ are defined as
\begin{align*}
	\prm p & = q \quad \text{where $\Prime (q)$ and $q\goesto^* p$} \\
	\cls p & = q \quad \text{where $\Closed(q)$ and $p\comesfrom^* q$} \enspace.
\end{align*}
%
Note that these are well-defined because $P$ is finite and $\goesto$ is acyclic, deterministic and injective.

\medskip\noindent
A pseudo-state space $\tupof{P,{\mapsto},{\goesto},{\up},\Tdepth}$ gives rise to a state space $\tupof{S,\trans{},\up,\Tdepth_S}$ where
%
\begin{itemize}
\item $s\in S$ if $s$ is a closed pseudo-state; i.e., $S=\cls P$;
\item $\cls p\trans a \cls{p'}$ for all $p\step a p'$; i.e., $s\trans a s'$ if $s\comesfrom^*\step a \goesto^* s'$;
\item $\up$ remains unchanged (noting that $p\up$ implies $\Closed(p)$ and hence $p\in S$);
\item $\Tdepth_S$ restricts $\Tdepth$ to $S$; i.e., $\Tdepth_S=\gensetof{(p,\Tdepth(p))}{p\in S}$.
\end{itemize}
%
Given a pseudo-state space $P$, a \emph{configuration} is a partial function $C:\cls P\pto P$. $C$ gives rise to a state space $\tupof{S_C,\trans{}_C,\up_C,\Tdepth_C}$ where
%
\begin{itemize}
\item $S_C=\dom C$;
\item For all $s,s'\in S_C$, $s\trans a s'$ if $C(s)\comesfrom^*\step a \goesto^* C(s')$;
\item For all $s\in S_C$, $s\up_C$ if $C(s)\up$;
\item For all $s\in S_C$, $\Tdepth_C(s)=\Tdepth(C(s))$.
\end{itemize}

\end{document}
